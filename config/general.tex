% ==========================
% UNIFIED TEX PREAMBLE
% ==========================

% ----------------------
% Coding and Language
% ----------------------
\usepackage[utf8]{inputenc}
\usepackage[T1]{fontenc}
\usepackage[main={lang}]{babel}

% ----------------------
% Typography and mathematics
% ----------------------
\usepackage{amsmath}
\usepackage{amssymb}
\usepackage{amsthm}
\usepackage{mathtools}
\usepackage{mathpartir}
{fontpkg}

% --------------------
% Symbol and utility packages
% --------------------
\usepackage{cancel}
\usepackage{textcomp}
\usepackage[mathscr]{euscript}
\usepackage[nointegrals]{wasysym}

% --------------------
% Extras
% --------------------
\usepackage{physics}
\usepackage{tikz}
\usepackage{tikz-cd}
\usetikzlibrary{decorations.markings}
\usetikzlibrary{calc}
\usepackage{graphicx}
\usepackage{geometry}
\usepackage{microtype}
\usepackage{fix-cm}
\usepackage{chemfig}
\usepackage{mhchem}
\usepackage{esvect}
\usepackage{nicematrix}
\usepackage{CJKutf8}
\usepackage{esint}
\usepackage{siunitx}
\usepackage{pgfplots}
\usepackage{circuitikz}
\usepackage{wrapfig}
% ----------------------
% Utilities and formatation
% ----------------------
\usepackage{mathrsfs}
\usepackage{indentfirst}
\usepackage{hyperref}
\usepackage{enumerate}
\usepackage{tabularx}
\usepackage{cancel}
\usepackage{comment}
\usepackage{csquotes}
\usepackage{hyphenat}
\usepackage{ellipsis}
\usepackage{paracol}
\usepackage[T1]{tipa}
\usepackage{xcolor}
\usepackage{fullwidth}
% ----------------------
% Logic and mathematics
% ----------------------
\usepackage{fitch}
\usepackage{proof}

\renewcommand*\land{\mathrel{\&}}
\newcommand*\Implies{\mathrel{\Rightarrow}}
\renewcommand*\implies{\mathrel{\rightarrow}}
\newcommand*\Iff{\mathrel{\Leftrightarrow}}
\renewcommand*\iff{\mathrel{\leftrightarrow}}
\newcommand*\ex{\mathrel{\downarrow}}

% ----------------------
% Customized math operators
% ----------------------
\DeclareMathOperator{\sgn}{sgn}
\DeclareMathOperator{\lcm}{lcm}
\DeclareMathOperator{\mmc}{mmc}
\DeclareMathOperator{\mdc}{mdc}
\DeclareMathOperator{\ifc}{if}
\let\Re\relax
\DeclareMathOperator{\Re}{Re}
\let\Im\relax
\DeclareMathOperator{\Im}{Im}

\renewcommand*\d{\mathrm{d}}

% Other math keybinds
\newcommand*{\mbb}[1]{\mathbb{#1}}
\newcommand*{\mcl}[1]{\mathcal{#1}}
\newcommand*{\mfr}[1]{\mathfrak{#1}}
\newcommand*{\func}[3]{#1:#2\to#3}
\newcommand{\vfunc}[5]{\func{#1}{#2}{#3},\quad#4\longmapsto#5}
\newcommand*{\floor}[1]{\left\lfloor#1\right\rfloor}
\newcommand*{\ceil}[1]{\left\lceil#1\right\rceil}


\newcommand*\df{\mathrm{df}}

\newcommand*\N{\mathbb{N}}
\newcommand*\Z{\mathbb{Z}}
\newcommand*\Q{\mathbb{Q}}
\newcommand*\R{\mathbb{R}}
\newcommand*\C{\mathbb{C}}
\newcommand*\F{\mathbb{F}}

\renewcommand*\O{\mathcal{O}}
\renewcommand*\emptyset{\varnothing}

% Some standard theorem definitions
\newtheorem{Theorem}{Theorem}
\newtheorem{Proposition}{Theorem}
\newtheorem{Lemma}[Theorem]{Lemma}
\newtheorem{Corollary}[Theorem]{Corollary}

\theoremstyle{definition}
\newtheorem{Definition}[Theorem]{Definition}


% ----------------------
% Always use math display
% ----------------------
\everymath{\displaystyle}

% ----------------------
% Configuração dos links
% ----------------------
\hypersetup{
    bookmarksnumbered=true,
    colorlinks=true,
    linkcolor=black,
    citecolor=black,
    urlcolor=blue,
}

% ----------------------
% Unicode Symbols (♠ ♣ ♥ ♦)
% ----------------------
\DeclareSymbolFont{extraup}{U}{zavm}{m}{n}
\DeclareMathSymbol{\spades}{\mathalpha}{extraup}{81}
\DeclareMathSymbol{\clubs}{\mathalpha}{extraup}{84}
\DeclareMathSymbol{\hearts}{\mathalpha}{extraup}{86}
\DeclareMathSymbol{\diamonds}{\mathalpha}{extraup}{87}

% ----------------------
% Correção do comportamento do \pmod*
% ----------------------
\makeatletter
\let\@@pmod\pmod
\DeclareRobustCommand{\pmod}{\@ifstar\@pmods\@@pmod}
\def\@pmods#1{\mkern4mu({\operator@font mod}\mkern 6mu#1)}
\makeatother

% ----------------------
% Melhor Overline (\widebar)
% ----------------------
\makeatletter
\let\save@mathaccent\mathaccent
\newcommand*\if@single[3]{%
  \setbox0\hbox{${\mathaccent"0362{#1}}^H$}%
  \setbox2\hbox{${\mathaccent"0362{\kern0pt#1}}^H$}%
  \ifdim\ht0=\ht2 #3\else #2\fi
}
\newcommand*\rel@kern[1]{\kern#1\dimexpr\macc@kerna}
\newcommand*\wideaccent[2]{\@ifnextchar^{{\wide@accent{#1}{#2}{0}}}{\wide@accent{#1}{#2}{1}}}
\newcommand*\wide@accent[3]{\if@single{#2}{\wide@accent@{#1}{#2}{#3}{1}}{\wide@accent@{#1}{#2}{#3}{2}}}
\newcommand*\wide@accent@[4]{%
  \begingroup
  \def\mathaccent##1##2{%
    \let\mathaccent\save@mathaccent
    \if#42 \let\macc@nucleus\first@char \fi
    \setbox\z@\hbox{$\macc@style{\macc@nucleus}_{}$}%
    \setbox\tw@\hbox{$\macc@style{\macc@nucleus}{}_{}$}%
    \dimen@\wd\tw@
    \advance\dimen@-\wd\z@
    \divide\dimen@ 3
    \@tempdima\wd\tw@
    \advance\@tempdima-\scriptspace
    \divide\@tempdima 10
    \advance\dimen@-\@tempdima
    \ifdim\dimen@>\z@ \dimen@0pt\fi
    \rel@kern{0.6}\kern-\dimen@
    \if#41
      #1{\rel@kern{-0.6}\kern\dimen@\macc@nucleus\rel@kern{0.4}\kern\dimen@}%
      \advance\dimen@0.4\dimexpr\macc@kerna
      \let\final@kern#3%
      \ifdim\dimen@<\z@ \let\final@kern1\fi
      \if\final@kern1 \kern-\dimen@\fi
    \else
      #1{\rel@kern{-0.6}\kern\dimen@#2}%
    \fi
  }%
  \macc@depth\@ne
  \let\math@bgroup\@empty \let\math@egroup\macc@set@skewchar
  \mathsurround\z@ \frozen@everymath{\mathgroup\macc@group\relax}%
  \macc@set@skewchar\relax
  \let\mathaccentV\macc@nested@a
  \if#41
    \macc@nested@a\relax111{#2}%
  \else
    \def\gobble@till@marker##1\endmarker{}%
    \futurelet\first@char\gobble@till@marker#2\endmarker
    \ifcat\noexpand\first@char A\else
      \def\first@char{}%
    \fi
    \macc@nested@a\relax111{\first@char}%
  \fi
  \endgroup
}
\newcommand\widebar{\wideaccent\overline}
\makeatother

